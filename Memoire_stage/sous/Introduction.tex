\chapter*{Introduction Génarale}
\markboth{Introduction Générale}{}
\addstarredchapter{Introduction Générale}

Dans un environnement mondial caractérisé par la démocratisation du web et
des nouvelles technologies digitales, plusieurs entreprises se voient devant
l’obligation de transformer leur mode de fonctionnement ainsi que leurs
processus de gestion pour s’aligner aux impératifs du digital. Cette
transformation, qui était considérée comme un choix dans les années
passées, s’impose de plus en plus puisqu’elle touche directement la
pérennité de l’entreprise.
\newline
Ce processus de transformation ne réside pas seulement dans l’adoption d’un ensemble
d’outils technologiques mais d’entreprendre un vaste chantier de mutation numérique et
organisationnel qui touche aux processus, ressources ainsi qu’à la manière de gérer les
données.
\newline
Cet axe relatif aux données, longtemps considéré comme l’axe le plus « stable » dans un
système d’information, est en train de subir des changements énormes surtout avec
l’avènement de nouveaux concepts comme le « Big Data » ainsi la prise de décision en temps
réel.
\newline
Le terme Big Data se réfère aux technologies et outils permettant de transformer les données
volumineuses, structurées, semi-structurées et non structurées en informations utiles pour la
prise de décision.
\newline
Par ailleurs, une des stratégies de gestion, qui a vu le jour après la crise de l’internet (bulle
internet) au début des années 2000, consiste à mieux se focaliser sur le cœur de l’activité des
entreprises et à externaliser toutes les autres activités secondaires comme les processus de
gestion des ressources humaines, de paie ainsi que la gestion du parc informatique.
\newline
C’est dans ce cadre bien précis que s’inscrit notre travail de mémoire de fin d’étude de
master effectué au sein de la société ECONOCOM qui propose à ses clients la possibilité
d’externaliser toute la gestion des infrastructures informatiques. Notre travail consiste donc à
la mise en place de solutions pour l’analyse, le traitement et la restitution des données
collectées de plusieurs sources de la part des différents équipements informatique installés sur
les sites des clients. Notre travail sera scindée en trois grandes étapes. 
\newline
Dans un premier temps, nous allons consolider et agréger les données des différentes sources, ensuite suivra l'étape de préparation et de nettoyage des données. Enfin les données consolidées nous permettra de prédire les matériels ou équipements  informatiques qui sont susceptibles de tomber en panne.
