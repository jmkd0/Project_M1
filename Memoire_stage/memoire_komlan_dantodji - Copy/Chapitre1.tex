\chapter{Le contexte de résolution du problème}
\chaptermark{Contexte}
\minitoc
\newpage
 
\section{Introduction}

La société ECONOCOM propose des solutions de transformation digitale comme la location des matériels informatiques pour faciliter les besoins des clients. Les données sont collectées et renseignées sur une plateforme cloud qui s'appelle ServiceNow. Nous avons décidé de prendre pour notre projet, des données de matériels informatiques (des ordinateurs).
Comme dit plus haut ces données concerne la location d'ordinateurs par les clients d'ECONOCOM.
Une fois les données collectées sur le ServiceNow, elles sont consolidées, nettoyées, préparées et enrichies.
La sortie finale est un fichier au format csv de 28 colonnes et 130 000 lignes.
\newline
Ci-dessous, une description des contenus de ces colonnes :
\newline
* ComputerSystemId : Identifiant du sytème d'exploitation de l'ordinateur 
\newline
* Model : Modèle de l'ordinateur
\newline
* DateProd : Date de fabrication de l'ordinateur
\newline
* IdleSeconde :  Température quand le processeur n'est pas exploité par seconde
\newline
* ActiveSeconde :  Le temps en seconde pendant lequel le processuer est exploité
\newline
* CPU :  Microprocesseur initial d'un ordinateur
\newline
* StandBySeconde : Temps d'attente du redémarrage du processus par seconde
\newline
* HibernateSeconde :  L'état de veille du système d'exploitation de l'ordinateur par seconde
\newline
* PhysicalMemory :  Taille Mémoire physique de l'ordinateur
\newline
* MaxPhysicalMemory : Taille maximale d ela mémoire physique de l'ordinateur 
\newline
* NbNormReboot : Nombre de redémarrage du sytème
\newline
* NbAbnReboot : Nombre de redémarrage du système
\newline
* TotalDataSentMo :  Taille totale des données envoyées par le système
\newline
* TotalDataReceivedMo :  Taille totale des données reçues par le système
\newline
* TotalDiskReadMo : Taille totale des données lues par le disque dur interne de l'ordinateur
\newline
* TotalDiskWriteMo : Taille totale des données écries sur le disque dur interne de l'ordinateur
\newline
* AvgDiskActivity : Temps moyen d'activité du disque interne de l'ordinateur 
\newline
* MaxDiskActivity : Temps maximum d'activité du disque dur interne de l'ordinateur 
\newline
* SLACompliancyTimeSeconde : Mesure de compilation du système par seconde


\section{Conclusion} 
En conclusion, l'objectif de notre sujet de mémoire est de faire de la prédiction sur des matériels informatiques afin de détecter ceux qui sont susceptibles de tomber en panne.


