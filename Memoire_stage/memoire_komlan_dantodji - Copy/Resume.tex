\chapter*{Résumé}
\markboth{Résumé}{}
\addstarredchapter{Résumé}
 
Dans le cadre de ce mémoire, j’ai effectué mon alternance au sein du service expertise technique et industrielle du groupe EDF. Mes missions effectuées au cours de cette alternance sont entre autres la mise en place des outils, algorithmes de traitement de données, utilisation des méthodes et techniques d’analyse de données, de fouille de données, d’apprentissage automatique sur des données de clients industriels. Ces données peuvent être des relevés au pas horaire, journaliers, mensuels ou autres de consommation électriques, météorologiques, données de température, quantités produites etc... Ces données une fois traitées permettent de produire aux clients industriels de l’entreprise des indicateurs clés de performance permettant d’expliquer le fonctionnement du site industriel sur une période donnée.

La première partie de mon alternance consistait tout d’abord à me familiariser avec l’entreprise ainsi que les métiers des membres de l’équipe expertise technique. Dans un second temps je devais me familiariser avec les outils numériques existants disponibles et utilisés par l’équipe expertise technique à travers des formations, des traitements et des analyses de données. Enfin la dernière partie fût consacrée à la mise en place de nouveaux outils numériques afin de faciliter l’utilisation de ces données.
Le thème retenu pour la validation de mon mémoire de Master 2 s’intitule : « Clustering de données de séries temporelles : Application aux données électriques ».




